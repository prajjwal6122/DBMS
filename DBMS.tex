\documentclass[12pt]{article}
\usepackage[english]{babel}
\usepackage{natbib}
\usepackage{url}
\usepackage[utf8x]{inputenc}
\usepackage{amsmath}
\usepackage{graphicx}
\graphicspath{{images/}}
\usepackage{parskip}
\usepackage{fancyhdr}
\usepackage{vmargin}
\setmarginsrb{3 cm}{2.5 cm}{3 cm}{2.5 cm}{1 cm}{1.5 cm}{1 cm}{1.5 cm}
\usepackage{hyperref}
\usepackage{color}
\usepackage[export]{adjustbox}
\title{\normalsize BIG DATA IN HEALTHCARE MANAGEMENT}								% Title
\author{19111040 Prajjwal sharma 6th SEM}								% Author
\date{\today}											% Date

\makeatletter
\let\thetitle\@title
\let\theauthor\@author
\let\thedate\@date
\makeatother

\pagestyle{fancy}
\fancyhf{}
\rhead{\theauthor}
\lhead{\thetitle}
\cfoot{\thepage}

\begin{document}

%%%%%%%%%%%%%%%%%%%%%%%%%%%%%%%%%%%%%%%%%%%%%%%%%%%%%%%%%%%%%%%%%%%%%%%%%%%%%%%%%%%%%%%%%

\begin{titlepage}
	\centering
    \vspace*{0.5 cm}
    \includegraphics[scale = 0.5]{logo.png}\\[1.0 cm]	% University Logo
    \textsc{\Large \newline NATIONAL INSTITUTE OF TECHNOLOGY RAIPUR}\\[2.0 cm]	% University Name
	\textsc{\Large BIOMEDICAL ENGINEERING}
	\\[0.5 cm]				% Course Code
	\rule{\linewidth}{0.2 mm} \\[0.4 cm]
	{ \huge \bfseries \thetitle}\\
	\rule{\linewidth}{0.2 mm} \\[1.5 cm]
	
	\begin{minipage}{0.5\textwidth}
		\begin{flushleft} \large
			\emph{Name:}\\
			Prajjwal Sharma\\
            5th SEMESTER\\
            19111040\\
			\end{flushleft}
			\end{minipage}~
			\begin{minipage}{0.4\textwidth}
            
			\begin{flushright} \large
			\emph{SUBJECT} \\
				\bf{DATABASE MANAGEMENT SYSTEM TERM PAPER}
		\end{flushright}
        
	\end{minipage}\\[2 cm]
	
	

    
    
    
	
\end{titlepage}

%%%%%%%%%%%%%%%%%%%%%%%%%%%%%%%%%%%%%%%%%%%%%%%%%%%%%%%%%%%%%%%%%%%%%%%%%%%%%%%%%%%%%%%%%




%%%%%%%%%%%%%%%%%%%%%%%%%%%%%%%%%%%%%%%%%%%%%%%%%%%%%%%%%%%%%%%%%%%%%%%%%%%%%%%%%%%%%%%%%

\section{Introduction}
The amount of information or data  healthcare organizations collect, manage and analyze has increased rapidly with advancements and integrations in technology. Technology, in turn, is changing the way the healthcare sector uses data. Advanced tools and software have been essential to the unprecedented growth of big data, making healthcare information easier and cheaper to store, access and use.Healthcare industry work with an tremendous amount of data.these data is needed to be stored examined and analysed in order to cure patients.the data can be of any type for example
\begin{itemize}
    \item \textbf{patinet records}
    \item \textbf{patient reports}
    \item \textbf{medicines}
    \item \textbf{equipment's lists} $etc$
\end{itemize}

\subsection{What is BIG DATA in Healthcare}
Big data in healthcare is a term used to describe massive volumes of information created by the adoption of digital technologies that collect patients' records and help in managing hospital performance, otherwise too large and complex for traditional technologies.

The application of big data analytics in healthcare has a lot of positive and also life-saving outcomes. In essence, big-style data refers to the vast quantities of information created by the digitization of everything, that gets consolidated and analyzed by specific technologies. Applied to healthcare, it will use specific health data of a population (or of a particular individual) and potentially help to prevent epidemics, cure disease, cut down costs, etc.


\end{document}